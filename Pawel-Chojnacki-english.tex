\documentclass[11pt,a4paper,sans]{moderncv}        

\moderncvstyle{classic}                             % style options are 'casual' (default), 'classic', 'oldstyle' and 'banking'
\renewcommand*{\cventry}[7][.25em]{%
  \cvitem[#1]{#2}{%
    {\bfseries#3}%
   \ifthenelse{\equal{#4}{}}{}{, {\slshape#4}}% I changed this line (with comma) ...
   % \ifthenelse{\equal{#4}{}}{}{ {\slshape#4}}% ... into this one (without comma).
    \ifthenelse{\equal{#5}{}}{}{, #5}%
    \ifthenelse{\equal{#6}{}}{}{ #6}%
    .\strut%
    \ifx&#7&%
      \else{\newline{}\begin{minipage}[t]{\linewidth}\small#7\end{minipage}}\fi}}
%%%
\moderncvcolor{blue}                               % color options 'blue' (default), 'orange', 'green', 'red', 'purple', 'grey' and 'black'
\nopagenumbers{}                                  % uncomment to suppress automatic page numbering for CVs longer than one page

\usepackage[utf8]{inputenc}                       % if you are not using xelatex ou lualatex, replace by the encoding you are using

\usepackage[scale=0.85]{geometry}
\setlength{\hintscolumnwidth}{3cm}                % if you want to change the width of the column with the dates

\name{Paweł}{Chojnacki}
\address{Warsaw, Poland}
\phone[mobile]{+48~730~461~681}                   % optional, remove / comment the line if not wanted
\email{pawel.marcin.chojnacki@gmail.com}                               % optional, remove / comment the line if not wanted
\homepage{github.com/Pawel-Marcin-Chojnacki}                         % optional, remove / comment the line if not wanted
\photo[64pt][0.4pt]{picture}                       % optional, remove / comment the line if not wanted; '64pt' is the height the picture must be resized to, 0.4pt is the thickness of the frame around it (put it to 0pt for no frame) and 'picture' is the name of the picture file
\begin{document}
\makecvtitle

\section{Professional summary}
% 
\cventry{}{}{}{}{Software engineer with four years of professional experience. At British Council I'm working as part of a team on maintaining and writing new features for exams registration platform. I'm full stack developer with biased to back-end technologies. During my previous work in BSB I improved database application performance tenfold. I was modernising an application from Silverlight tech to ASP.NET MVC. Currently I'm gaining experience in .NET Core 2.2, Angular}{}

\section{Professional experience}
\cventry{February 2017 -- July 2018}{Programmer}{Bazy i Systemy Bankowe Sp. z o.o}{Bydgoszcz}{}{I was responsible for the development and maintenance of SPERT (C\#/.NET, MySQL) system used in most of the Polish banking institutions. I was leading order for functional changes in the taxation project ecoTAX (C\#/.NET, MsSQL) for the Marshal's Office. Couple of weeks before leaving the company I was in a team developing OZIN (ASP.NET MVC, MsSQL, JavaScript), a new product of the National Bank of Poland.}
\cventry{March 2016 -- February 2017}{Junior programmer}{Bazy i Systemy Bankowe Sp. z o.o}{Bydgoszcz}{}{I was responsible for the design and implementation of the new features in the anti-terrorism and money laundering software. After the end of the project, I worked in a team designated to write SWPF, a new software for financial planning of the National Bank of Poland.}

\section{Projects}
\cventry{Climate Observer}{.NET Framework 4.7.1, WPF, Quartz.NET, Moq, SQLite, Entity Framework, xUnit, AutoFake, LiveCharts}{}{A Windows Service collects weather information using OpenWeatherMap for a list of selected cities. Desktop application allows user to show collected data in a from of interactive charts}{}{}
\cventry{Kopra}{Windows Phone 8.1 SDK, MVVM pattern}{}{Mobile application that supports the use of social lending services. It's targeted for regular investors. Using WebAPI of \textit{Kokos.pl}, investor has critical information on his/her smartphone at the right time}{}{}
%\cventry{PhotoAlbum}{C\#/.NET}{Library}{making use of an EXIF in jpg files. Enables creation of photo albums, tagging photos, adding description, title, suggesting tags from previously used. Because all information is stored in file itself, everything that user writes is portable and platform independent}{}{}
%\cventry{Home Media Manager}{AngularJS}{}{Application to manage movies in personal library on mobile devices. Rotten tomatoes API was used to compare movies ratings}{}{}
\cventry{FERS}{C\#/.NET, LiveSDK API, Google API, Phonegap, Quartz.NET}{Fast and Easy Reservation System (team project)}{A complete solution for a room booking. The system consists of a web application for booking and managing staff accounts, a Windows Service for syncing calendars, and a mobile Android app to manage reservations.}{I was responsible for creating synchronization service that pushes changes to third party calendar services, updates security keys, etc. I also created an app for emplyees to view room reservations}{}

\section{Skills}
\cventry{\textsc{C\#/.NET}}{ASP.NET MVC, WCF, WPF, MVVM, Entity Framework, Moq, xUnit}{}{Technology used in my professional work. I direct my eductaion around .NET stack}{}{}
\cventry{\textsc{SQL}}{Microsoft SQL, Oracle PL/SQL, MySQL}{}{I migrated data between databases running on Microsoft and Oracle solutions. I wrote procedures aggregating large sets of data and software parametrization scripts for several of Polish banks. I was responsible for optimizing client process execution time}{}{}
\cventry{\textsc{JavaScript}}{Tensorflow.js}{}{I took part in writing the front end of a few applications using jQuery and Bootstrap. I am currently learning JavaScript by developing a side project using Tensorflow.js and Angular}{}{}
\cventry{\textsc{Python}}{Tensorflow, Keras, request, PyWin, pil}{}{I use Python for Machine Learning, data analysis, testing concepts and scripting tedious manual tasks}{}{}
\cventry{\textsc{Design Patterns}}{Facade, Factory Method, Abstract Factory, CQRS, Command, Dependency Injection}{}{}{}{}
\cventry{\textsc{Git}}{}{}{}{I use the GitHub repositories on a daily basis, I also create my own repositories there}{}
\cventry{\textsc{SVN}}{}{}{}{I was using it in the professional work}{}
\cventry{\textsc{Visual Studio}}{}{}{}{I've worked on all versions across Visual Studio 2008 -- 2017 (with Resharper extension), Code, VS for Mac}{}
\cventry{\textsc{Other}}{Agile software development, SCRUM, SOLID principles}{}{}{}{}

\section{Education}
\cventry{2017 -- 2018}{Computer Science}{Nicolaus Copernicus University,}{}{\textit{Masters}}{}
\cventry{2012 -- 2017}{Computer Science}{Nicolaus Copernicus University,}{}{\textit{engineering degree}}{}  % arguments 3 to 6 can be left empty
\cventry{2014 -- 2015}{Mobile apps}{Universitat de Vic,}{}{\textit{Erasmus+ exchange}}{}  % arguments 3 to 6 can be left empty
\cventry{2008 -- 2012}{IT technician}{ZSMEiE, Toruń}{}{}{}

\section{Certificates}
\cventry{04.2018}{Udemy}{Natural Language Processing with Deep Learning in Python}{}{}{}
\cventry{02.2015 -- 09.2015}{Microsoft Approved Course}{ITA-113 Programming Mobile Devices}{}{}{}

\section{Languages}
\cvitemwithcomment{polish}{Mothertongue}{}
\cvitemwithcomment{english}{Advanced}{Conversationally fluent}

\section{Hobbies}
\cvitem{}{learning new technologies -- currently .NET Core 2.1, Deep Learning}
\cvitem{}{long-distance running -- I run 10--15 km. I ran the marathon in 2017}
\cvitem{}{Latin dance}

\section{Additional assets}
\cventry{}{Participation in the Academic Championships in Team Programming 2013 (AMPPZ)}{Programmer in a 3-person team representing NCU 2013}{}{}{}{}
\cventry{}{Participation in the Association for Computing Machinery International Collegiate Programming Contest Central Europe Regional Contest 2012}{ACM ICPC CERC}{Programmer in a 3-person team representing UMK}{}{}
\cventry{}{Participation in the Erasmus+ student exchange program}{}{}{}{}
\cventry{}{Member of the Student Research Group}{}{}{}{}

\cfoot{\footnotesize{I hereby give consent for my personal data included in my application to be processed for the purposes of the recruitment process under the Personal Data Protection Act as of 29 August 1997, consolidated text: Journal of Laws 2016, item 922 as amended.}}

\end{document}
